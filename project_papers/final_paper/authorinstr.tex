\documentclass{article}
\usepackage{bnaic}


%% if your are not using LaTeX2e use instead
%% \documentstyle[bnaic]{article}

%% begin document with title, author and affiliations

\title{\textbf{\huge A Reinforcement Learning Approach for Solving Chess Endgames}%
}
\author{Zacharias Georgiou \affila \and
    Evangelos Karountzos \affila \and
    Matthia Sabatelli \affila \and
    Yaroslav Shkarupa \affila}
\date{\affila\ \textit{Department of Artificial Intelligence\\
  Nijenborgh 4,
  9747 AG Groningen\\
  The Netherlands }\\
 }

\pagestyle{empty}

\begin{document}
\ttl
\thispagestyle{empty}


\begin{abstract}
\noindent
This is the abstract of my paper. Please start the first paragraph of your abstract with a \verb+\noindent+ command.
This is the abstract of my paper. Please start the first paragraph of your abstract with a \verb+\noindent+ command.
This is the abstract of my paper. Please start the first paragraph of your abstract with a \verb+\noindent+ command.
\end{abstract}


\section{Introduction}

Since the moment computers and intelligent machines have become able to compute complicated calculations, one way of testing these abilities, has been identified in making them play games. Nothing better then games, in fact, are able to test if a machine is performing in an intelligent way or not, by choosing correct and appropriate actions, its resulting behaviour may lead it to a win or not. This general concept finds concrete evidence already during the 18th century when the Hungarian inventor and engineer Wolfgang Von Kempelen, designed the first chess playing automaton FIXME REF known under the name of "The Turk". This machine was designed with the purpose of playing chess in a completely autonomous way, without any human help it was, in fact, supposed to beat almost all of its human opponents. This early and rough robot turned out to become a celebrity, not only in Europe but also in the United States, its inventor toured in fact between the old and the new continent, making his creation play against very famous opponents of that time. One of the most important games that may be found in chess literature was played by the Turk against Napoleon Bonaparte, where the french general lost very quickly in not even 25 moves. However, this machine turned out to be a deceit, Von Kempelen's invention wasn't in fact able to play autonomously at all, inside the machine there was place for a human person that was able to see the chessboard and move the robot with some commands without being seen from the audience. Although "The Turk" can't be defined as an authentic autonomous player since it turned out to be a fake, it still has the merit to be the first example in which machine and humans challenge each other by playing chess.\\
The idea of building a machine able to beat the best human players at chess has inspired several mathematicians and computer scientists during the years and the moment they were all waiting for, arrived in May of 1997 REF. Deep Blue, a chess playing software developed by the American company IBM, was able to defeat the current chess world champion Garry Kasparov in a six-game match with the score of 3.5 against 2.5. It was the first time that a chess A.I. was able to defeat the best chess player in the world, this event is considered as an absolute breakthrough and represents the end of an era for chess players and the start of a new one for A.I. scientists.\\
However, although A.I. has reached such important results as the one presented previously chess still remains a      

add NP complexity, unsolved...approximations...
            


\section{Related Research}


\section{Our Approach}


\section{Results}


\section{Discussion and Conclusion}



\bibliographystyle{plain}
\bibliography{mybibfile}



\end{document}








